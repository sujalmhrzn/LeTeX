\chapter{INTRODUCTION}
% (20% of Proposal Length)
\pagenumbering{arabic}

% Introduction: (20\% of Report Length)


\section{Introduction}
The Build Wizard project aims to develop an innovative online platform that assists PC builders in Nepal in selecting compatible components, comparing prices, and engaging with a vibrant community. This  provides a comprehensive background for the project, outlining the need, objectives, and potential impact.The PC building industry in Nepal has witnessed significant growth in recent years, fueled by increasing demand for custom-built PCs. However, PC builders often face challenges in selecting compatible components and comparing prices across multiple retailers. The absence of a localized platform catering specifically to the Nepalese market creates a gap that the Build Wizard project intends to address.
\section{Problem Statement}

The process of selecting and assembling compatible components for custom PC builds poses significant challenges for individuals in the rapidly evolving PC building industry. With a plethora of PC components available and constant technological advancements, users face difficulties in navigating the complex landscape, often resulting in compatibility issues, time-consuming troubleshooting, and suboptimal hardware choices.\\\\
Thus,a comprehensive solution is required to address these problems, providing a user-friendly interface, an extensive and up-to-date component database, real-time pricing information, compatibility verification, and a vibrant community for sharing expertise and support. By tackling these challenges, users will be empowered to confidently select and assemble compatible PC components, optimize their budget, and create high-performance custom PC builds tailored to their specific needs and preferences
\section{Objectives}
\begin{itemize}
    \item The objective of the project is to create a comprehensive and localized solution that simplifies the PC building process.
\end{itemize}
\section{Scope}
% Scope and limitation

The scope of the Build Wizard project in Nepal would encompass providing a localized version of the platform tailored to the needs and preferences of Nepalese users. The project would aim to address the specific challenges faced by PC builders in Nepal and offer a comprehensive solution for component selection, compatibility verification and pricing comparison.\\Localized Component Database: The project would involve compiling a comprehensive and up-to-date database of PC components available in the Nepalese market. This would include processors, motherboards, graphics cards, memory modules, storage drives, power supplies, cooling solutions, peripherals, and other relevant hardware.\\
Pricing Integration: The platform would integrate with local retailers and online marketplaces to provide real-time pricing information from Nepalese vendors. This functionality would enable users to compare prices, find the best deals, and optimize their budget while making informed purchasing decisions.\\Vendor Partnerships: Collaborating with local PC component retailers and manufacturers would be crucial to ensure accurate and up-to-date information, pricing integration, and availability of components. Building partnerships with these entities would enhance the platform's credibility and provide users with a seamless experience.\\Education and Resource Hub: The platform could also feature educational content, tutorials, and resources specific to the Nepalese context, empowering users with the necessary knowledge and skills for successful PC builds.
\section{Limitation}
\begin{itemize}
    \item This system doesn't have offline customizing.
    \item The platform may not fully cater to users with limited technical expertise.
    \item Any inaccuracies or outdated data could impact compatibility checks or pricing information
 \end{itemize}
\section{Potential Applications}
 A PC part picker project has several potential applications that can benefit both individual users and the broader computing community. Some of these applications include:
\begin{itemize}
    \item Custom PC Building: The primary application is assisting users in selecting compatible components for building their own custom PCs. It streamlines the process, ensuring that the selected components work well together, and helps users optimize performance according to their needs. 
    \item Gaming Enthusiasts: Gamers often require high-performance systems. This project can aid them in building gaming rigs that meet their specific requirements, taking into account factors like graphics card capabilities, cooling solutions, and more.
    \item Educational Use: The platform can serve as an educational tool, helping students learn about different PC components, their specifications, and how they work together. It could be integrated into technology courses or workshops.
    \item DIY PC Building Workshops: Institutions or organizations offering DIY PC building workshops can utilize this tool to simplify the process for participants.
\end{itemize}
\section{Originality of Project}
The originality of a PC part picker project lies in how it innovatively addresses challenges and provides value that sets it apart from existing solutions.
\begin{itemize}
    \item Unique Feature Set: This project offers novel features not commonly found in other PC part pickers, such as advanced compatibility algorithms.
    \item Educational Aspect: This project not only helps users select components but also educates them about the technology behind each component, it can be unique in promoting learning alongside building.
\end{itemize}
\section{Report Organisation}
The material in this project report is organised into seven chapters. After this introductory chapter introduces the problem topic this research tries to address, chapter 2 contains the literature review of vital and relevant publications, pointing toward a notable research gap. Chapter 3 describes the methodology for the implementation of this project. Chapter 4 provides an overview of what has been accomplished. Chapter 5 contains some crucial discussions on the used model and methods. Chapter 6 mentions pathways for future research direction for the same problem or in the same domain. Chapter 7 concludes the project shortly, mentioning the accomplishment and comparing it with the main objectives.